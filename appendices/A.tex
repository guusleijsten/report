%Appendix C.

%TODO: mention that I left out CMakeFiles and LLVMBuildInfo for simplicity of the directory tree.
%\section{File Structure}
This appendix provides pseudo code of the pass that exploits explicit datapaths from Chapter \ref{sec:expl_bp_impl}. First of all, the children of each node in the dominator tree are sorted on reverse post-order tree traversel. Then each node and the functions \texttt{EnterScope} and \texttt{ProcessBlock} are called.\\

%TODO: add more introducing text, and explain pseudo code with text.
\begin{algorithm}[H]
 \KwData{PerformEBA}
 \KwResult{Explicit bypass allocation is performed}
 Sort dominator tree on reverse post order traversal\;
 
 Node = depthFirst(DominatorTree.root())\;
 \While{node}{
  block = Node.block()\;
  EnterScope(block)\;
  ProcessBlock(block)\;
  Node = next(Node)\;
 }
 \caption{Outer function called by \texttt{RunOnMachineFunction}.}
\end{algorithm}

\texttt{EnterScope} sets up the pipeline state by taking the bypass state from the most-frequently executed predecessor block. Subsequently, it modifies other predecessor blocks to match their resulting pipeline state when it is required by a bypass.\\

\begin{algorithm}[H]
 \KwData{EnterScope}
 \KwResult{Pipeline state is setup and conflicts on predecessors has been resolved}
 \eIf{isLoop}{
  \For{block in loop}{
   AnalyzeBlock(state, block)\;
  }
  setState(state)\;
 }{
  \eIf{preds.size() $>$ 1}{
   setState(MostFrequentlyExecutedPred.getState())\;
   checkConflicts(preds)\;
  }{
   \eIf{preds.size $==$ 1}{
    setState(pred.getState())\;
   }{
       initState()\;
   }
  }
 }
 \caption{First function called by \texttt{PerformEBA}.}
\end{algorithm}



\newpage



Where \texttt{AnalyzeBlock} goes through a basic block, and updates the pipeline state model accordingly. The procedure calls to \texttt{setState} and \texttt{getState} set the pipeline state model to a given state, or queries the pipeline state model of a basic block. The following pseudo code shows functionality of \texttt{ProcessBlock}, which is similar to \texttt{AnalyzeBlock}, but also allocates explicit bypasses that it finds.

\begin{algorithm}[H]
 \KwData{ProcessBlock}
 \KwResult{Explicit bypass allocation is performed on a basic block}
 \For{instruction in block}{
  allocateBypasses(instruction)\;
  
  insertIntoPipeline(instruction)\;
  register instruction in pipeline state model\;
  \If{end cycle}{
   propagate pipeline state\;
  }
 }
 \caption{Second function called by \texttt{PerformEBA}.}
\end{algorithm}

\begin{algorithm}[H]
 \KwData{allocateBypasses}
 \KwResult{Explicit bypass allocation is performed on an instruction}
 \For{operand in instruction}{
  match = matchOperandInPipeline()\;
  \If{match}{
   bypass operand\;
  }
  \If{match $\And$ isKill}{
   avoid store\;
  }
 }
 \caption{Inner function called by \texttt{ProcessBlock}.}
\end{algorithm}